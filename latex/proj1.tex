% !TeX spellcheck = en_GB
\documentclass[12pt,fleqn]{article}

\usepackage[danish]{babel}
\usepackage{SpeedyGonzales}
\usepackage{MediocreMike}
%\usepackage{Blastoise}

\usepackage{float}
\usepackage[caption = false]{subfig}
\title{}
\author{Asger Schultz}
\date{\today}

\fancypagestyle{plain}
{
	\fancyhf{}
	\rfoot{Side \thepage{} af \pageref{LastPage}}
	\renewcommand{\headrulewidth}{0pt}
}
\pagestyle{fancy}
\fancyhf{}
\lhead{Asger Schultz}
\chead{}
\rhead{}
\rfoot{Side \thepage{} af \pageref{LastPage}}

\graphicspath{{Billeder/}}
\linespread{1.15}


%\numberwithin{equation}{section}
%\numberwithin{footnote}{section}
%\numberwithin{figure}{section}
%\numberwithin{table}{section}

\begin{document}

\maketitle
%\thispagestyle{fancy}
%\tableofcontents
--Identifying people from movement curve\\
--Are arm movements of different objects significantly different\\
--Trains, evaluates binary classification tree and 3-nearest neighbour classification to identify people. McNemars test: Find that 3-nearest neighbours is significantly than class. tree and that both are significantly better than baseline.\\
--Using four-way ANOVA, it is found that there is significant difference in arm movements for different experiments.
\tableofcontents
\newpage 


\section{Introduction}

-- Briefly introduce the background \& setting of the problem, as well as the aim of the report. Furthermore, you could give a very short description of the analysis that will be applied.


\section{Data}
The data consists of 16 experiments where ten right-handed people had to move a cylinder over another cylinder.
The size and weight of the cylinders varied over the experiments, and each person had to perform the movement of each experiment 10 times.
Thus, the data was structured in a $ 16\times 10\times 10 $ grid, where the first axis is the experiment, the second the person, and the third the repetition.
For every such repetition, the $ x, y, $ and $ z $ coordinates of the movement where recorded 100 times, such that the data for each repetition was a $ 100\times 3 $ matrix.
The data had been preprocessed such that every curve was of the same length.\\
\\
For our purposes, it was benificial to consider each recording of $ x, y, $ and $ z $ a set of three features, so we ravelled the data such that each repetition had $ 300 $ features with one observation each.
For the first part of the report, we investigated only experiment 4, so we had a total of 100 observations.
The target variable was the person who performed the movement, so the goal of our machine learning models was a 10-class classification.
\begin{figure}[H]
		
	\centering
	\subfloat{
		\includegraphics[width=.5\linewidth]{p1_3d_example1}
	}
	\subfloat{5
		\centering
		\includegraphics[width=.5\linewidth]{p1_3d_example2}
	}
\end{figure}


\section{Methods and analysis}


\subsection{Machine Learning Task: Classification}
-- High dimensional 10-class classification 
\paragraph{Models}
---Binary classification tree: Hunt's algorithm\\
---3NN: The K value was chosen before based on data dimensionality\\
-- Baseline: 10\pro\\  

\paragraph{Performance evaluation}
--Leave-one-out crossvalidation\\
--McNemar's test\\
-- Note, for McNemar's test: Baseline makes a difference. Was chosen as random guesses, as no single class should be preferred.
\subsection{Test of experiment effect}

-- Introduce ANOVA: Model assumptions
-- Introduce factors



\section{Results}
\begin{figure}[H]
	\centering
	\includegraphics[width=.7\linewidth]{mcnemar_results}
\end{figure}
\begin{figure}[H]
\centering
\includegraphics[width=.7\linewidth]{p1_anova}
\end{figure}
\begin{figure}[H]
	\centering
	\includegraphics[width=.7\linewidth]{p1_anova_summay}
\end{figure}


Yes, experiment is significant.\\
	
-- All experiments are (at 5\pro) significantly different from the control experiment which has been set to reference

\section{Discussion}
\subsection{Classification of persons}
--Plot: Seems nonlinear (parabolic data)\\
-- Plot: For humans: Easy difference in \(y\)-coordinate between first repetitions of the two initial test subjects.

\subsection{Test of difference in experiments}
-- Expectation: Yes, significant difference as 15 different obtacle avoidance tests + control is considered\\
-- Surprise: Repetition is also significant though\\
-- Which coordinate it is explains the largest part of data variability but experiment no. comes in second\\
-- Low residual variance when compared to within-group variance\\
-- Model assumptions: Plot\\

\appendix
\section{hwo}
\begin{figure}[H]
	
	\centering
	\subfloat{
		\includegraphics[width=.5\linewidth]{p1_example}
	}
	\subfloat{5
		\centering
		\includegraphics[width=.5\linewidth]{p1_example2}
	}
\end{figure}
\end{document}

















